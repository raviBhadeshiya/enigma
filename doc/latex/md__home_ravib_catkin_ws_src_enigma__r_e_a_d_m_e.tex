\href{https://travis-ci.org/raviBhadeshiya/enigma}{\tt } \mbox{[}\mbox{]}(L\+I\+C\+E\+N\+SE) \section*{enigma}

One of the big tech topics of 2017 is automation−whether and how robots can replace or augmentwork done by humans. In order to do the kind of work a human security guard would normally do,the robot should use cameras, sensors, navigation equipment, and electric motors−all packed intoits dome-\/shaped body with a big rechargeable battery and a computer. The autonomous guidancesystem allows the robot to store its cruise route and move automatically without an operator. It is based on the input video camera and other sensors; allowing the U\+GV to follow the path accurately,detect and avoid the obstacles. The package present real time indoor security device we call Enigma! The robotics system using Turtl\+Bot to perform a surveillance of a virtual environment generated by 3D Gazebo simulator. This system performs an indoor surveillance byperforming S\+L\+AM and video processing, in order to ensure that any possible ananomly is detected by the color of an object such as Red or Green colored cylinder.

The system will 3\+D-\/map an unknown environment. While 3\+D-\/mapping the unknown environment, the bot will perform patrol by running the detection node to find the ananomly such as red-\/colored cylinder object. There is be custom world with red and green colored cylinders. The detection algorithm is very naive such as transform image to H\+SV space, then thresholding and region detection for the object (red colored cylinder).Then object count is publish on the \char`\"{}\textbackslash{}detection\char`\"{} topic and user will get notified by info message. Please check below\+:



\subparagraph*{Output with 3\+D-\/mapping}



\subsection*{Dependency}


\begin{DoxyItemize}
\item \href{http://wiki.ros.org/ROS/Installation}{\tt R\+OS Kinetic} on Ubuntu 16.\+04
\item \href{http://gazebosim.org/}{\tt Gazebo}
\item \href{http://wiki.ros.org/Robots/TurtleBot}{\tt Turtle\+Bot}
\item \href{http://wiki.ros.org/octomap}{\tt octomap}
\end{DoxyItemize}

\#\# Standard install via command-\/line 
\begin{DoxyCode}
$ mkdir -p ~/catkin\_ws/src
$ cd ~/catkin\_ws/
$ catkin\_make
$ source devel/setup.bash
$ cd src/
$ git clone --recursive https://github.com/raviBhadeshiya/enigma.git
$ cd ..
$ catkin\_make
\end{DoxyCode}
 $>$N\+O\+TE\+: Missing dependency can be added via following command but highly encourage to install sepratly!


\begin{DoxyCode}
~/catkin\_ws$ rosdep install --from-paths src --ignore-src --rosdistro kinetic -y
\end{DoxyCode}
 However for octomap installation follow the tutorial\+:\href{http://wiki.ros.org/octomap}{\tt link} 

 \subsection*{Execution}

The following package can ber start with single launch file as follows\+: 
\begin{DoxyCode}
$ roslaunch enigma enigma.launch
\end{DoxyCode}
 $>$Arguments\+: record\+:=true for recording rosbag file and world\+:=enigma/map/final\+\_\+world\+\_\+1 for testing custom word file.

$>$Note\+: roslaunch will automatically start roscore if it detects that it is not already running. \subparagraph*{Package Interface\+:}

The following package use the rosservice to intract with package. \subparagraph*{Start/\+Stop Robot Motion}

To call this service, open a new terminal and type\+:


\begin{DoxyCode}
$ rosservice call /robotSwitch true
\end{DoxyCode}
 Argument true will cause robot to start motion and false will cause robot to stop the motion. $>$Note\+: After roslaunch if robot is not moving, call this service to start motion. \subparagraph*{Change Robot Speed}

If you want to change the speed of the robot,then open a new terminal and make new service call\+:


\begin{DoxyCode}
$ rosservice call /robotSpeed 0.8
\end{DoxyCode}
 Argument will set the robot max speed to 0.\+8 but you can pass any valid speed between 0 to 1.\+0 \subparagraph*{Start/\+Stop \hyperlink{class_detection}{Detection}}

To call this service, similarly open a new terminal and type\+: 
\begin{DoxyCode}
$ rosservice call /detectionSwitch true
\end{DoxyCode}
 Argument true will cause detection to start detection and false will cause to stop. If \hyperlink{class_detection}{Detection} topic is not being publishe, then call this service to start detection. \subparagraph*{Saving the Octomap}

The octomap\+\_\+server loads a 3D map (as Octree-\/based Octo\+Map) and distributes it to other nodes in a compact binary format. It also allows to incrementally build 3D Octo\+Maps, and provides map saving in the node octomap\+\_\+saver. 
\begin{DoxyCode}
$ rosrun octomap\_server octopmap\_saver -f enigma\_map.ot
\end{DoxyCode}


\paragraph*{Running the node separately}

$>$Caution\+: Make sure turtlebot-\/gazebo is running.

Running nodes requires you have a R\+OS core started. A R\+OS core is a collection of nodes and programs that are pre-\/requisites of a R\+O\+S-\/based system. You must have a roscore running in order for R\+OS nodes to communicate. Open a new shell, and type\+: 
\begin{DoxyCode}
$ roscore
\end{DoxyCode}
 \subparagraph*{Node}

The \hyperlink{class_enigma}{Enigma} node can be run by opening new shell, and type\+: 
\begin{DoxyCode}
$ rosrun enigma enigma\_node
\end{DoxyCode}
 and for detection node can be run by opening new shell, and type\+: 
\begin{DoxyCode}
$ rosrun enigma detection\_node
\end{DoxyCode}
 

 \subsection*{Testing}

There are two main unit tests in this module\+:testing for enigma and detection. To run all testcase of this module\+: 
\begin{DoxyCode}
$ cd catkin\_ws
$ catkin\_make run\_tests
\end{DoxyCode}
 Or using rostest, type this command in new terminal\+: 
\begin{DoxyCode}
$ rostest enigma enigma\_check.test
$ rostest enigma detection\_check.test
\end{DoxyCode}
 

 \subsubsection*{Solo Iterative Process}

Solo Iterative process was used for developing this package and it can be observed that estimates were improved over time. For detailed spreadsheet\+: \href{https://docs.google.com/spreadsheets/d/10tGs0astZB6bFPMXlLJwByrlTDJi1ZNfbLnQGZDo5Xo/edit?usp=sharing}{\tt }

For sprint planning\+: \href{https://docs.google.com/document/d/1hJ8q-_5HhWBHmOfXV9d_nZg0-gOeu8520EBP83TKBbI/edit?usp=sharing}{\tt Click Here}

\subsubsection*{Presentation}

Slides \+: \href{https://docs.google.com/presentation/d/1KM6IbHGB0xqtuu0a0-CgijcBst5EShQxvior-kgVUcg/edit?usp=sharing}{\tt click here} 